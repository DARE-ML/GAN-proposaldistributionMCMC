%%%%%%%%%%%%%%%%%%%%%%%%%%%%%%%%%%%%%%%%%%%%%%%%%%%%%%%%%%%%%%%%%%%%%%%
%
%  A small sample UNSW Honours Thesis file.
%  Any questions to Ian Doust i.doust@unsw.edu.au
%
% Edited CSG 11.9.2015, use some of Gery's ideas for front matter; add a conclusion chapter.
%%%%%%%%%%%%%%%%%%%%%%%%%%%%%%%%%%%%%%%%%%%%%%%%%%%%%%%%%%%%%%%%%%%%%%%
%
%  The first part pulls in a UNSW Thesis class file.  This one is
%  slightly nonstandard and has been set up to do a couple of
%  things automatically
%
 
\documentclass[honours,12pt]{unswthesis}
\linespread{1}
\usepackage{amsfonts}
\usepackage{amssymb}
\usepackage{amsthm}
\usepackage{latexsym,amsmath}
\usepackage{graphicx}
\usepackage{afterpage}

%%%%%%%%%%%%%%%%%%%%%%%%%%%%%%%%%%%%%%%%%%%%%%%%%%%%%%%%%%%%%%%%%
%
%  The following are some simple LaTeX macros to give some
%  commonly used letters in funny fonts. You may need more or less of
%  these
%
\newcommand{\R}{\mathbb{R}}
\newcommand{\Q}{\mathbb{Q}}
\newcommand{\C}{\mathbb{C}}
\newcommand{\N}{\mathbb{N}}
\newcommand{\F}{\mathbb{F}}
\newcommand{\PP}{\mathbb{P}}
\newcommand{\T}{\mathbb{T}}
\newcommand{\Z}{\mathbb{Z}}
\newcommand{\B}{\mathfrak{B}}
\newcommand{\BB}{\mathcal{B}}
\newcommand{\M}{\mathfrak{M}}
\newcommand{\X}{\mathfrak{X}}
\newcommand{\Y}{\mathfrak{Y}}
\newcommand{\CC}{\mathcal{C}}
\newcommand{\E}{\mathbb{E}}
\newcommand{\cP}{\mathcal{P}}
\newcommand{\cS}{\mathcal{S}}
\newcommand{\A}{\mathcal{A}}
\newcommand{\ZZ}{\mathcal{Z}}
%%%%%%%%%%%%%%%%%%%%%%%%%%%%%%%%%%%%%%%%%%%%%%%%%%%%%%%%%%%%%%%%%%%%%
%
% The following are much more esoteric commands that I have left in
% so that this file still processes. Use or delete as you see fit
%
\newcommand{\bv}[1]{\mbox{BV($#1$)}}
\newcommand{\comb}[2]{\left(\!\!\!\begin{array}{c}#1\\#2\end{array}\!\!\!\right)
}
\newcommand{\Lat}{{\rm Lat}}
\newcommand{\var}{\mathop{\rm var}}
\newcommand{\Pt}{{\mathcal P}}
\def\tr(#1){{\rm trace}(#1)}
\def\Exp(#1){{\mathbb E}(#1)}
\def\Exps(#1){{\mathbb E}\sparen(#1)}
\newcommand{\floor}[1]{\left\lfloor #1 \right\rfloor}
\newcommand{\ceil}[1]{\left\lceil #1 \right\rceil}
\newcommand{\hatt}[1]{\widehat #1}
\newcommand{\modeq}[3]{#1 \equiv #2 \,(\text{mod}\, #3)}
\newcommand{\rmod}{\,\mathrm{mod}\,}
\newcommand{\p}{\hphantom{+}}
\newcommand{\vect}[1]{\mbox{\boldmath $ #1 $}}
\newcommand{\reff}[2]{\ref{#1}.\ref{#2}}
\newcommand{\psum}[2]{\sum_{#1}^{#2}\!\!\!'\,\,}
\newcommand{\bin}[2]{\left( \begin{array}{@{}c@{}}
				#1 \\ #2
			\end{array}\right)	}
%
%  Macros - some of these are in plain TeX (gasp!)
%
\newcommand{\be}{($\beta$)}
\newcommand{\eqp}{\mathrel{{=}_p}}
\newcommand{\ltp}{\mathrel{{\prec}_p}}
\newcommand{\lep}{\mathrel{{\preceq}_p}}
\def\brack#1{\left \{ #1 \right \}}
\def\bul{$\bullet$\ }
\def\cl{{\rm cl}}
\let\del=\partial
\def\enditem{\par\smallskip\noindent}
\def\implies{\Rightarrow}
\def\inpr#1,#2{\t \hbox{\langle #1 , #2 \rangle} \t}
\def\ip<#1,#2>{\langle #1,#2 \rangle}
\def\lp{\ell^p}
\def\maxb#1{\max \brack{#1}}
\def\minb#1{\min \brack{#1}}
\def\mod#1{\left \vert #1 \right \vert}
\def\norm#1{\left \Vert #1 \right \Vert}
\def\paren(#1){\left( #1 \right)}
\def\qed{\hfill \hbox{$\Box$} \smallskip}
\def\sbrack#1{\Bigl \{ #1 \Bigr \} }
\def\ssbrack#1{ \{ #1 \} }
\def\smod#1{\Bigl \vert #1 \Bigr \vert}
\def\smmod#1{\bigl \vert #1 \bigr \vert}
\def\ssmod#1{\vert #1 \vert}
\def\sspmod#1{\vert\, #1 \, \vert}
\def\snorm#1{\Bigl \Vert #1 \Bigr \Vert}
\def\ssnorm#1{\Vert #1 \Vert}
\def\sparen(#1){\Bigl ( #1 \Bigr )}

\newcommand\blankpage{%
    \null
    \thispagestyle{empty}%
    \addtocounter{page}{-1}%
    \newpage}

%%%%%%%%%%%%%%%%%%%%%%%%%%%%%%%%%%%%%%%%%%%%%%%%%%%%%%%%%%%%%%
%
% These environments allow you to get nice numbered headings
%  for your Theorems, Definitions etc.  
%
%  Environments
%
%%%%%%%%%%%%%%%%%%%%%%%%%%%%%%%

\newtheorem{theorem}{Theorem}[section]
\newtheorem{lemma}[theorem]{Lemma}
\newtheorem{proposition}[theorem]{Proposition}
\newtheorem{corollary}[theorem]{Corollary}
\newtheorem{conjecture}[theorem]{Conjecture}
\newtheorem{definition}[theorem]{Definition}
\newtheorem{example}[theorem]{Example}
\newtheorem{remark}[theorem]{Remark}
\newtheorem{question}[theorem]{Question}
\newtheorem{notation}[theorem]{Notation}
\numberwithin{equation}{section}

%%%%%%%%%%%%%%%%%%%%%%%%%%%%%%%%%%%%%%%%%%%%%%%%%%%%%%%%%%%%%%%%%%
%
%  If you've got some funny special words that LaTeX might not
% hyphenate properly, you can give it a helping hand:
%
\hyphenation{Mar-cin-kie-wicz Rade-macher}

%%%%%%%%%%%%%%%%%%%%%%%%%%%%%%%%%%%%%%%%%%%%%%%%%%%%%%%%%%%%%%%%%%
% 
% OK...Now we get to some actual input.  The first part sets up
% the title etc that will appear on the front page
%
%%%%%%%%%%%%%%%%%%%%%%%%%%%%%%%%%%%%%%%%%%%%%%%%%%%%%%%%%%%%%%%%%

\title{Surrogated Assisted Bayesian Neural Network for Geological Models}

\authornameonly{Sean Luo}

\author{\Authornameonly\\{\bigskip}Supervisor: Professor Rohitash Chandra, Professor Richard Scalzo}

\copyrightfalse
\figurespagefalse
\tablespagefalse

%%%%%%%%%%%%%%%%%%%%%%%%%%%%%%%%%%%%%%%%%%%%%%%%%%%%%%%%%%%%%%%%%
%
%  And now the document begins
%  The \beforepreface and \afterpreface commands puts the
%  contents page etc in
%
%%%%%%%%%%%%%%%%%%%%%%%%%%%%%%%%%%%%%%%%%%%%%%%%%%%%%%%%%%%%%%%%%%

\begin{document}

\beforepreface

\afterpage{\blankpage}

% plagiarism

\prefacesection{Plagiarism statement}

\vskip 10pc \noindent I declare that this thesis is my
own work, except where acknowledged, and has not been submitted for
academic credit elsewhere. 

\vskip 2pc  \noindent I acknowledge that the assessor of this
thesis may, for the purpose of assessing it:
\begin{itemize}
\item Reproduce it and provide a copy to another member of the University; and/or,
\item Communicate a copy of it to a plagiarism checking service (which may then retain a copy of it on its database for the purpose of future plagiarism checking).
\end{itemize}

\vskip 2pc \noindent I certify that I have read and understood the University Rules in
respect of Student Academic Misconduct, and am aware of any potential plagiarism penalties which may 
apply.\vspace{24pt}

\vskip 2pc \noindent By signing 
this declaration I am
agreeing to the statements and conditions above.
\vskip 2pc \noindent
Signed: \rule{7cm}{0.25pt} \hfill Date: \rule{4cm}{0.25pt} \newline
\vskip 1pc

\afterpage{\blankpage}

% Acknowledgements are optional


\prefacesection{Acknowledgements}

{\bigskip}By far the greatest thanks must go to my supervisor for
the guidance, care and support they provided. 
 
{\bigskip} Thanks to my family for supporting me to advance in higher education.

{\bigskip\noindent} Thanks go to Fred Flintstone and Robert Taggart for allowing his thesis
style to be shamelessly copied.

{\bigskip\bigskip\bigskip\noindent} Sean Luo, Day Month Year.

\afterpage{\blankpage}

% Abstract

\prefacesection{Abstract}

This thesis is an investigation of the ?????????

\afterpage{\blankpage}


\afterpreface

%%%%%%%%%%%%%%%%%%%%%%%%%%%%%%%%%%%%%%%%%%%%%%%%%%%%%%%%%%%%%%%%%%
%
% Now we can start on the first chapter
% Within chapters we have sections, subsections and so forth
%
%%%%%%%%%%%%%%%%%%%%%%%%%%%%%%%%%%%%%%%%%%%%%%%%%%%%%%%%%%%%%%%%%%

\afterpage{\blankpage}

\chapter{Introduction}\label{intro}

In the past decade there has being some progression in the field of bayesian deep learning. Bayesian deep learning offers an intrinsic way to ensemble models that helps to quantify the posterior uncertainty in model parameters and prediction. Applying tradition bayesian inference techniques such as Monte Carlo sampling in deep neural networks have multiple challenges, mainly revolving around computational expenses due to the evaluation speed of large dataset, exponentially increasing rejection rate in MCMC sampler as the parameter space dimensions grows with large model, difficulty to evaluate uncertainty in realtime data, efficient proposal convergence and others. Variational bayesian inference methods are faster than sampling, but it does not offer an exact approximation of the target distribution. 


%% other challenges to be include via further paper reading

\noindent Ideas explored: 

Pure Bayesian:

\begin{enumerate}
\item Using stochastic mini batches by welling 2011

\item Using langevin dynamic gradient as proposal distribution, by chandra 

\item Using parallel tempering to speed up exploration and exploitation by chandra \cite{LDPTBNN}

\item Using surrogate models to simulate the posterior likelihood $p(D|\theta)$ to speed up evaluation, by chandra. \cite{SAPTBNN}

\item Surrogate models are selected from variantional inference category, including, GAN, VAE, and Normaling Flow.

\item Modification to frequentist models that turns into bayesian:

\item Using dropout as intrinsic ensemble (bayesian?).

\item Using stochastic weight averaging withe gaussian noise (SWAG) - Wilson et al
\end{enumerate}

This paper proposes to have the surrogate to generate weight and its corresponding numerator and denominator during accept reject to avoid all the extra forward pass and backward loss compare to normal DNN training. The input would be a window of past model parameters and a random vector from normal distrbution, The output would be a new set of model parameters, the corresponding numerator and denominator of accept reject step.

\section{Generative Models}

VAE is a shrinkage training model that has two shrinker that trains to a latent space of mean and variance of normal distribution.
from the two latent vector it then trains to simulate a proper input.

GAN have discriminator and generator, generator samples from standard normal and maps it via a network to output space, discriminator compare the overall generated result to dataset. The training is conducted via a maximizing loss for discriminator and minimizing loss on generator.

Normalizing Flow is composed of multiple inversable layers that helps to provide a tractable posterior likelihood on weights. The target loss function for continuous case (we care for continuous case only as we plan to use it to generate the weights) is $L(D) \approxeq \frac{1}{N}\sum_{i=1}^N - \log(p_\theta(\tilde{x}^{(i)}))+c$. Where $\tilde{x}^{(i)} = x^{i}+u$ with $u\sim U(0,a)$ and $c=-M\times \log{a}$, $a$ is dependent on data discretization and $M$ is the dimension of $x$. This loss function encourages the network to learn to distribute data across the domain of the standard normal distribution. Specifically the GLOW model proposed by OpenAI \cite{openai2018glow} uses 1x1 convolution to boost training performance.


\section{ Notes:}

Swag : approximates local loss distribution (posterior?) using standard normal momentum distribution.
Propose -> Accept Reject: replacing forward pass for proposal 
Surrogate by Chandra et al: evaluate proposal and spit out its posterior likelihood
    i.e. 
    
    $min(1, pi(w*|x)/pi(wi|x) *  q(wi|w*)/q(w*|wi) )$
    
    $min(1, p(x|w*)/p(x|wi) * p(w*|x)/p(wi|x) *  q(wi|w*)/q(w*|wi) )$

    $p(w|x)$ and $q(wi|w*)$ are both easy to compute, the first is a loss, the second is proposal distribution (using tractable distribution like normal that s easy to evaluate)    
    $p(x|w)$ requires forward pass, so we evaluate it using surrogate




\noindent Packages of interest include pymc, pyro, pytorch, (maybe  sydney-machine-learning/pingala?)


%%%%%%%%%%%%%%%%%%%%%%%%%%%%%%%%%%%%%

\chapter{Sampling}\label{samp}





%%%%%%%%%%%%%%%%%%%%%%%%%%%%%%%%%%%%%%%%%%%%%%%%%%%%%%%%%%%%%%%%%%%%

\chapter{Method}\label{meth}




%%%%%%%%%%%%%%%%
\chapter{Experiment}\label{expe}


\chapter{Analysis}\label{anal}


\chapter{Discussion}\label{disc}

\chapter{Conclusion}\label{conc}



%%%%%%%%%%%%%%%%%%%%%%%%%%%%%%%%%%%%%%%%%%%%%%%%%%%%%%%%%%%%%%%%%%%%%%%%%%

\clearpage

\addcontentsline{toc}{chapter}{References}

\LaTeX{} \cite{latex2e} is a set of macros built atop \TeX{} \cite{texbook}.
\bibliographystyle{plain} % We choose the "plain" reference style
\bibliography{refs} % Entries are in the refs.bib file

\end{document}





